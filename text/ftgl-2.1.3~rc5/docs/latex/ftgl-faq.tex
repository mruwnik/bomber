\section{FAQ}\label{ftgl-faq_faq}
\subsection{When I try to compile \%FTGL it complains about a missing file from the include: \#include $<$ft2build.h$>$}\label{ftgl-faq_faq1}
FTGL relies on FreeType 2 for opening and decoding font files. This include is the main include for FreeType. You will need to download Freetype 2 and install it. Then make sure that the FTGL project that you are using points to your FreeType installation.\subsection{Is it possible to map a font to a \char`\"{}unit\char`\"{} size? My application relies on the fonts being a certain \char`\"{}physical\char`\"{} height (in OpenGL coordinate space) rather than a point size in display space. Any thoughts/suggestions?}\label{ftgl-faq_faq2}
We can do anything:) It would be easy to allow you to set the size in pixels, though I'm not sure this is what you want. Setting the size to 'OpenGL units' may be a bit harder. What does 1.0 in opengl space mean and how does that relate to point size? For one person it might mean scaling the font up, for someone else it may mean scaling down. Plus bitmaps and pixmaps have a pixel to pixel relationship that you can't change.

Here's some guidelines for vector and texture fonts. Take note that I say 'should' a lot :)

\begin{itemize}
\item One point in pixel space maps to 1 unit in OpenGL space, so a glyph that is 18 points high should be 18.0 units high.\end{itemize}


\begin{itemize}
\item If you set an ortho projection to the window size and draw a glyph it's screen size should be the correct physical size ie a 72 point glyph on a 72dpi screen will be 1 inch high. Also if you set a perspective projection that maps 0.0 in the z axis to screen size you will get the same eg. 

\begin{Code}\begin{verbatim}gluPerspective(90, window_height / 2 , small_number, large_number);
\end{verbatim}
\end{Code}

 So basically it all depends on your projection matrix. Obviously you can use glScale but I understand if you don't want to.\end{itemize}


Couple of extra things to note:

\begin{itemize}
\item The quality of vector glyphs will not change when you change the size, ie. a really small polygon glyph up close will look exactly the same as a big one from far away. They both contain the same amount of data. This doesn't apply to texture fonts.\end{itemize}


\begin{itemize}
\item Secondly, there is a bug in the advance/kerning code that will cause ugliness at really small point sizes. This is because the advance and kerning use ints so an advance of 0.4 will become zero. If this is going to be a probelm, I can fix this.\end{itemize}


Early on I did a lot of head scratching over the OpenGL unit to font size thing because when I was first integrating FTGL into my engine the fonts weren't the size I was expecting. I was tempted to build in some scaling but I decided doing nothing was the best approach because you can't please everyone. Plus it's 'correct' as it is. 